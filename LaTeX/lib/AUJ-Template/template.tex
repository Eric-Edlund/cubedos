\documentclass{AUJarticle}
\usepackage[cmex10]{amsmath}
\usepackage[utf8x]{inputenc}
\usepackage[nocompress]{cite}
\usepackage{graphicx, multirow, booktabs, color, listings}
\usepackage{balance}

\graphicspath{{figures/}}

\pagestyle{empty}

\pdfinfo{ /Title (Template for Ada User Journal)
  /Author (An Author, B. Another, Y. Other)
  /Keywords (Ada) }

\newcommand{\todo}{\textcolor{red}}

\hyphenation{}

\setcounter{page}{3}

\begin{document}

\title{Template for Ada User Journal}

\addauthor{A. N. Author, B. Another}
{The Ada Research Institute, Lovelace Road, 
  Oxbridge OB5 1AZ.; Tel: +44 123 987 6543} 
  {\{ana, ban\}@ada.org}

\addauthor{Y. Other}
{Ada Rules, All Languages Road, 
  The World} 
  {yother@adarules.org}

\issuev{35}
\issuen{1}
\issued{March 2014}

\shortauthor{A. N. Author, B. Another, Y. Other}
\shorttitle{Template for Ada User Journal}

\thispagestyle{plain}

\maketitle

\begin{abstract}

This template gives guidance to authors on the preparation of papers for Ada User Journal. It is
written in the style of such a paper. Thus an abstract can be written here. The style is
Abstract and is Times 11/12 italic and justified. The abstract is optional and if present can
optionally be followed by a keyword list using the same style.

Keywords: template, journal, Ada.

\end{abstract}

\section{Introduction}

Measurements are expressed using the traditional units of points and picas for typesetting.
There are 12 points to a pica and 6 picas to an inch.

The general style for text is that all paragraphs are justified but not indented and are
followed by a 6pt blank. Normal text is set in Times 10/12. That is with a 10 point font and two
points of leading giving a line spacing of 12 points. The style for this paragraph is TP Text
para.

The two letter mnemonic starting the style name gives the shortcut keys which can be used to
apply it. Thus to apply Text para, place the cursor within the paragraph, hold down control and
shift and then type TP. Widow/orphan control is on although this does mean that 3-line
paragraphs cannot be split and may cause unusual pagination.


\section{Page layout}
The printed page width is 42 picas. Papers are set in two justified columns with a 2 pica gutter
giving 20p per column.

Left and right margins on A4 are thus 3.8 picas each. The top and bottom margins are 5 picas.
Headers and footers will be set as in this sample but may be omitted on submitted material. The
first page just has the page number. Subsequent even (left) pages (verso pages to use
typesetter's jargon) have the paper title truncated to fit on one line if necessary and odd
(right) pages have the author's name.

The last page must be set with equal columns.

\section{Headings and subheadings}

Headings are set in 13 pt Times bold with 6pt before and 6pt after; the style is MH Main
heading. The heading of the abstract (if any) is the same. Note that since the last paragraph of
the preceding section will be followed by a 6pt blank this means that the effective space before
is 12pt.

Only the first word of a heading should be capitalized unless it is a proper noun or some other
word which is normally capitalized.

\subsection{Subheadings}

Subheadings are set in 11pt Times bold with 3pt before and 3pt after; the style is SH Sub
heading. Further levels of heading are left to the author. It is suggested that they use the
subheading style but with italic rather than bold.

\subsection{Numbering of headings}

Headings and subheadings need not be numbered. But if they are then main headings should have
their number without any trailing point and with three spaces before the heading text. Similarly
for subheadings as illustrated here. It is recommended that further levels of headings not be
numbered. The abstract (if any) is never numbered. The remaining sections and subsections of
this template are not numbered by way of illustration.

\section{Title and authors}

Title information is set across the whole page but unjustified. 

The title is in Times 24/30; major words have their initial letter capitalized. The title is
followed by the author's name and affiliation. The name is in Times 12pt bold italic and the
affiliation is in Times 11pt italic but not bold. The title is followed by 12pt, the author by
3pt and the affiliation by 6pt.

The affiliation should include postal address; it can also include email and telephone number if
the author doesn't mind lots of junk mail and calls.

In the case of multiple authors with the same affiliation information, put the several authors
in the one paragraph.

If, however, multiple authors have different affiliation information then repeat author and
affiliation paragraphs as necessary.

The last affiliation paragraph should be followed by a blank paragraph (as in this template) in
order to create the appropriate amount of white space before the text proper.

\section{Initials and punctuation}
Author's names can be given with first name in full (preferred) or with initials. If using
initials then please follow initials with a point . For example Fred Scringe; F. L. Scringe;
Fred L. Scringe; are all acceptable. Academic titles are omitted.

\section{Program text}
Program text should be displayed in the running text if possible. Arial 9/12 should be used.

\begin{lstlisting}%

with Text_IO;
procedure Main is
begin  -- this is a comment
  Text_IO.Put("Hello World");
end;

\end{lstlisting}

\subsection{Blank lines in programs}

Long sections of program benefit from judicious use of blank lines. Thus we might have

\begin{lstlisting}%

package Stacks is
  type Stack is private;

  procedure Push(S: Stack; X: Integer);

  function Pop(S: Stack) return Integer;

private
  ...  -- the private part
end Stacks;
\end{lstlisting}

However, the full blank lines look crude and accordingly a sequence of such text should be set
using distinct paragraphs with style PP Program para which then gives a 6pt blank thus

\begin{lstlisting}%

package Stacks is
  type Stack is private;
  procedure Push(S: Stack; X: Integer);
  function Pop(S: Stack) return Integer;
private
  ...  -- the private part
end Stacks;
\end{lstlisting}


If a program sample is long or demands very long lines then it can be set across both columns
and treated as a figure (see the guidelines in the next section). In such a case it might be
considered desirable to use Courier for such program samples in order to match the appearance of
a genuine program even though program text in the discussion is set in Arial. If this is done
then perhaps reserved words should not be bold and comments should not be in italic and blank
lines should be in full (so just treat the whole sample as one paragraph).

\section{Figures and captions}

The positioning of figures (and tables, pictures etc) is left to the discretion of the author
but the following guidelines should be considered. They should be on the same page as the point
of reference wherever possible. They should either fit within one column or extend right across
both columns. If they extend across both columns then they should be at the top of the page. If
they fit within one column then they should be at the top of the column containing the reference
if possible but it is permissible to place them embedded within the text if that seems more
natural.

Captions should be in Times 9pt bold and numbered Figure 1, Table 6 and so on. They should be
centred under the item concerned as in the following example.

\section{Punctuation and spelling}

Use logical punctuation. Place stops in the logical position with respect to parentheses and
quotations. Thus write: The following string has six characters "Rabbit". Do not write: The
following string has six characters "Rabbit." (Some readers will be amazed that one might even
consider the latter style but it was prevalent in Britain in the early part of the 20th century
and still lingers in the US.) Also go easy on stops, write US rather than U.S.

It is simpler to use feet and inch marks (' and ") rather than printers quotes and apostrophes
which are not in the Latin-1 character set.

Spelling is left to the author but when in doubt use the form preferred by the Oxford English
dictionary (thus optimize rather than optimise).

Apply hyphenation with care. This template has been set without hyphenation. The trouble with
automatic hyphenation is that the algorithms in most word processors are naive and can hyphenate
words at inappropriate points unrelated to the logical division of the word. Nevertheless, 20p
columns such as these may benefit from hyphenation.

\section{Citations and references}

References should be collected at the end and numbered in alphabetical order, eg:
\cite{Higgon1994, Dirac1930, Dijkstra1968,Keller2000}. Citations within text should take the
form: Smith [3] states that programming is boring. Multiple references may be grouped thus: but
recent research [7, 9, 10] shows that many find it exciting.

A sentence should never start with just a citation number (or indeed with any other number).

This template concludes with a sample reference section. The heading is a normal main heading.
Individual references have the style References.

\bibliographystyle{ieeetr}
\bibliography{bibliography}
\balance

\end{document}
