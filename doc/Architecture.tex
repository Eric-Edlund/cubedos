
\chapter{Architecture}
\label{chapt:architecture}

In this section we describe the high level architecture of CubedOS. This includes both the
architecture of the system and the architecture of the development environment.

\section{Overview}
\label{sec:overview}

To understand the context of the CubedOS architecture, it is useful to compare the architecture
of a CubedOS-based application with that of a more traditional application. Since CubedOS abides
by the restrictions of the \textbf{Core.Ravenscar} requirement, we must compare CubedOS with
other Ravenscar-based approaches.

Figure~\ref{fig:traditional-architecture} shows an example application using Ravenscar tasking.
Tasks, which must all be library level infinite loops, are shown in open circles and labeled as
$T_1$ through $T_4$. Tasks communicate with each other via protected objects, shown as solid
circles and labeled as $PO_1$ through $PO_4$.

\begin{figure}[tbhp]
  \center
  \scalebox{0.50}{\includegraphics*{Figure-Traditional.pdf}}
  \caption{Traditional Ravenscar-based Architecture}
  \label{fig:traditional-architecture}
\end{figure}

Arrows from a sending task to a protected object indicate calls to a protected procedure to
install information into the protected object. Arrows from a protected object to a receiving
task indicate calls to an entry in the protected object used to pick up information previously
stored in the object. Entry calls will block if no information is yet available but protected
procedure calls do not block.

Ravenscar requires that protected objects have at most one entry and that at most one task can
be queued on that entry. In CubedOS applications each protected object is serviced by exactly
one task. This ensures that two tasks will never accidentally be queued on the protected
object's entry \emph{provided the mapping from task (actually module, see below) to protected
  object is truly bijective}. In the figure this means only one arrow can emanate from a
protected object. Furthermore, in CubedOS applications each task (actually module) can have only
one arrow incident upon it. Note that Ravenscar applications do not suffer from this limitation.
However, in any case, multiple arrows can lead to a protected object, since it is permitted for
many tasks to call the same protected procedure or for there to be multiple protected procedures
in a given protected object.

In the example application of Figure~\ref{fig:traditional-architecture}, tasks $T_1$ and $T_3$
call protected procedures in two different protected objects. This presents no problems since
protected procedures never block, allowing a task to call both procedures in a timely manner.
However, task $T_3$ calls two entries, as allowed by Ravenscar but not CubedOS, one in $PO_1$
and another in $PO_2$. Since entry calls can block, this means the task might get suspended on
one of the calls leaving the other protected object without service for an extended time. The
application needs to either be written so that will never happen or be such that it doesn't
matter if it does.

There are several advantages to the traditional organization:

\begin{itemize}
\item The protected objects can be tuned to transmit only the information needed so the overhead
  can be kept minimal.
\item The parameters of the protected procedures and entries specify the precise types of the
  data transfered so compile-time type safety is provided.
\item The communication patterns of the application are known statically, facilitating analysis.
\end{itemize}

However the traditional architecture also includes some disadvantages:

\begin{itemize}
\item The protected objects must all be custom designed and individually implemented, creating a
  burden for the application developer.
\item The communication patterns are relatively inflexible. Changing them requires overhauling
  the application.
\end{itemize}

The CubedOS version of the sample application has an architecture as shown in
Figure~\ref{fig:cubedos-architecture}. In this case CubedOS provides the communication
infrastructure as an array of general purpose, protected mailbox objects. CubedOS
\textit{modules} communicate by sending messages to the receiver module's mailbox. The messages
are unstructured octet streams, and thus completely generic. Each module has exactly one mailbox
associated with it and contains a single task dedicated to servicing that mailbox, creating a
one-to-one relationship between modules and mailboxes. A module's mailbox servicing task
extracts messages from the mailbox, decodes the messages, and then acts on the messages.

Note that CubedOS modules are allowed to have additional internal tasks, if required, as long as
the usual Ravenscar rules are obeyed. These internal tasks do not communicate directly with
other modules.

The communication connections shown in Figure~\ref{fig:cubedos-architecture} are the same as
those shown in Figure~\ref{fig:traditional-architecture} except that the two communication paths
from $T_1$ to $T_3$ are combined into a single path going through one mailbox.

\begin{figure}[tbhp]
  \center
  \scalebox{0.40}{\includegraphics*{Figure-CubedOS.pdf}}
  \caption{CubedOS-based Architecture}
  \label{fig:cubedos-architecture}
\end{figure}

CubedOS relieves the application developer of the problem of creating the communications
infrastructure manually. Adding new message types is simplified with the help of a tool,
\texttt{Merc}, stored in a different repository of the CubeSat Laboratory GitHub. In addition to
providing basic, bounded mailboxes, CubedOS also provides other services such as message
priorities and multiple sending modalities (for example, best effort versus guaranteed
delivery). Many of these additional services would be tedious to provide on a case-by-case basis
following the traditional architecture. CubedOS also allows any module to potentially send a
message to any other module. Thus the communication paths in the running application are very
flexible.

Although the CubedOS architecture supports only point-to-point message passing, CubedOS comes
with a library module that supports a publish/subscribe discipline. The module allows multiple
channels to be created to which other modules can subscribe. Publisher modules can then send
messages to one or more channels. Since the messages themselves are unstructured octet streams,
the publish/subscribe module can handle them generically without being modified to account for
new message types.

Every CubedOS module has a statically assigned ID number. Messages sent from a module include
the ID number of the sender. This allows a server module to return reply messages without
statically knowing its clients. Thus server modules can be written as part of a general purpose
library and used without modification in a variety of applications. We have started compiling a
registry of ``well known'' module IDs for common services, such as file handling. This allows
CubedOS module libraries to make use of well known services and remain reusable.

We are also defining standard message interfaces to certain services, such as file handling,
that third party modules could implement. This allows modules to use a service without knowing
which specific implementation backs that service.

However, CubedOS's architecture also carries some significant disadvantages as well:

\begin{itemize}
\item All mailboxes must have the same size since they are stored in an array. This arises
  because the \textbf{Core.SPARK} requirement forbids the use of pointers. Consequently some
  mailboxes will be larger than necessary, wasting space.

\item All messages have the same type.\footnote{\textbf{Core.SPARK}'s prohibition on pointers
    prevents more flexibile designs.} This means the size of a message must be large enough to
  satisfy the needs of every module. As a result, in many cases messages will be larger and
  slower to copy than necessary.

  The common message type also requires that typed information sent from one module to another
  be encoded into a raw octet format when sent, and decoded back into specifically typed data
  when received. The encoding and decoding increases the runtime overhead of message passing and
  reduces type safety. Modules must defend themselves, at runtime, from malformed or
  inappropriate messages, causing certain errors that were compile-time errors in the
  traditional architecture to now be runtime errors. This is exactly counter to the general
  goals of high integrity system development.

\item In order to return reply messages, the mailboxes must be addressable at runtime using
  module ID numbers. \pet{More should be said about the issue of replies. How does the
    traditional model handle them?} Accessing a statically named mailbox isn't general enough.
  As a result, the precise communication paths used by the system cannot easily be determined
  statically.

  In particular, since \SPARK\ does not attempt to track information flow through individual
  array elements, it is necessary to manually justify certain \SPARK\ flow messages. Although
  the architecture of CubedOS ensures that there is a one-to-one correspondence between a module
  and its mailbox. The tools don't know this and the spurious flow messages they produce must be
  suppressed.
\end{itemize}

The details of CubedOS mitigate, to some degree, the problems above. For example, the mailbox
array is actually instantiated from a generic unit by the application developer. This allows the
developer to tune the sizes of the mailboxes, and the messages they contain, to the
application's needs. CubedOS does not attempt to provide a one-size-fits-all mailbox array that
will be satisfactory to all applications.

Also every well behaved CubedOS module should contain an |API| package with subprograms for
encoding and decoding messages. This package is generated by the \texttt{Merc} tool. The
parameters to these subprograms correspond to the parameters of the protected procedures and
entries in the traditional architecture, and provide much of the same type safety. However,
using the API subprograms is not enforced by the compiler. It is also possible to accidentally
send a message to the wrong mailbox. Thus modules still need to include runtime error checking
to detect and handle these problems.

So far we have described two extremes: a traditional approach that does not use CubedOS at all,
and an approach that entirely relies on CubedOS. However, hybrid approaches are also possible.
Figure~\ref{fig:hybrid-architecture} shows a combination of several CubedOS mailboxes and a
hand-made, optimized protected object to mediate communication from $T_3$ to $T_4$.

\begin{figure}[tbhp]
  \center
  \scalebox{0.40}{\includegraphics*{Figure-Hybrid.pdf}}
  \caption{Hybrid Architecture}
  \label{fig:hybrid-architecture}
\end{figure}

This provides the best of both worlds. The simplicity and flexibility of CubedOS can be used
where it makes sense to do so, and yet critical communications can still be optimized if the
results of profiling indicate a need. In Figure~\ref{fig:hybrid-architecture} task $T_4$ can't
be reached by CubedOS messages. The hand-made protected object creates a degree of isolation
that can also simplify analysis as compared to a pure CubedOS system. In effect, from CubedOS's
point of view, $T_4$ is an internal task of module \#3 and thus part of module \#3.

It is also possible to instantiate the CubedOS message manager multiple times in the same
application, effectively creating multiple communication domains using separate mailbox arrays.
Figure~\ref{fig:multi-domain} shows an example of where $T_4$ is in a separate domain from the
other modules (because it receives from a mailbox that is separate from the others). In a more
realistic example, the second communication domain would contain more than one module.

\begin{figure}[tbhp]
  \center
  \scalebox{0.40}{\includegraphics*{Figure-MultiDomain.pdf}}
  \caption{Multiple Communication Domains}
  \label{fig:multi-domain}
\end{figure}

This approach allows the CubedOS infrastructure to be used for easy development while still
partitioning the system into semi-independent sections. For example, the sizes of the mailboxes
and of the messages used in each communication domain need not be the same. The parts of the
application that require large messages could be grouped into a domain separate from the parts
that only require small messages.

Notice in Figure~\ref{fig:multi-domain} tasks $T_3$ and $T_4$ send messages into multiple
domains. This is, of course, sometimes necessary if the domains are going to interact. In such a
scenario each communication domain would regard the other domain as a collection of internal
tasks that is part of the module(s) that send messages into the other domain. We currently have
very little experience with systems designed in this way.


\section{Message Encoding}
\label{section-message-encoding}

CubedOS mailboxes store messages as unstructured octet arrays. This allows a general purpose
mailbox package to store and manipulate messages of any type. Unfortunately this also requires
that well structured, well typed message information be encoded to raw octets before being
placed in a mailbox and then decoded after being retrieved from a mailbox.

The CubedOS convention is to use External Data Representation (XDR) encoded messages. XDR is a
well known standard \cite{rfc-4506} that is also simple and has low overhead. We have defined an
extension to XDR that allows \SPARK's constrained scalar subtypes to be represented and that
allows limited contracts to be expressed. The tool \texttt{Merc} compiles a high level
description of a message into message encoding and decoding subprograms. Our tool is written in
Scala and is not verified, but its output is subject to the same \SPARK\ analysis as the rest of
the system. In our case it is easier to prove the output of \texttt{Merc} than it is to prove
the correctness of \texttt{Merc} itself.

The use of \texttt{Merc} mitigates some of CubedOS's disadvantages. The developer need not
manually write the tedious and repetitive encoding and decoding subprograms. Furthermore, those
subprograms have well-typed parameters thus shielding the programer from the inherent lack of
type safety in the mailboxes themselves.

To illustrate CubedOS message handling, consider the following short example of a message
definition file that is acceptable to \texttt{Merc}.

\begin{verbatim}
enum Series_Type { One_Shot, Periodic };

typedef unsigned int Module_ID range 1 .. 16;
typedef unsigned int Series_ID_Type range 1 .. 10000;

message struct {
    Module_ID      Sender;
    Time_Span      Tick_Interval;
    Series_Type    Request_Type;
    Series_ID_Type Series_ID;
} Relative_Request_Message;
\end{verbatim}

This file introduces several types following the usual syntax of XDR interface definitions. The
syntax is extended, however, to allow the programmer to include constrained ranges on the scalar
type definitions in a style that is normal for Ada. The message itself is described as a
structure containing various components in the usual way. The reserved word |message| prefixed
to the structure definition, an extension to XDR, alerts \texttt{Merc} that it needs to generate
encoding and decoding subprograms for that structure. Other structures serving only as message
components can also be defined.

\texttt{Merc} has built-in knowledge of certain Ada private types such as |Time_Span| (from the
|Ada.Real_Time| package). Private types need special handling since their internal structure
can't be accessed directly from the encoding and decoding subprograms. There is currently no
mechanism in \texttt{Merc} to solve this problem in the general case.

Each message type has an ID number that is unique across the application. This is required to
distinguish valid messages from invalid ones. \texttt{Merc} assigns these message type ID
numbers automatically. The encoding subprogram installs the ID number into the message as a
hidden component. The decoding subprogram checks the message type ID number and returns a
failure status if it is invalid. A CubedOS module extracts a message from its mailbox and then
applies the relevant decoder subprograms to that message until one succeeds. If all decoder
subprograms fail the message either has an unexpected type (for example it was accidentally sent
to the wrong mailbox) or the message is malformed in some way. The module must then make a
runtime decision about how to handle that situation.

\texttt{Merc} is a work in progress. We intend to ultimately support as much of the XDR standard
as we can including, for example, variable length arrays and discriminated unions.

\section{Source Code Organization}
\label{sec:code-organization}

In this section we describe the layout and organization of the source code of CubedOS. If you
intend to work on the development of the software you should read this section carefully.

\begin{itemize}

\item \filename{src}. The main source folder. The code in this folder is hardware independent;
  it calls into various packages to interact with low level devices. All the code in this folder
  will fly in space and must be SPARK.

  Note that this folder is taken as the root folder for all relative paths used in this document
  from now on. This folder will now be referenced as the ``root folder'' or as ``.'' depending
  on the context.

\item \filename{Check}. This folder contains the main file and all supporting code for the AUnit
  based unit test program. Hardware specific packages, if they are implemented at all, contain
  stubs that receive data from the higher level packages and present that data to the test
  program for evaluation. The file \filename{check.adb} is the test program's main file. None of
  the code in this folder needs to be SPARK since it will not fly in space.

\item \filename{Mock}. This folder contains software simulations of the low level hardware
  devices. Using these simulations it is possible to build CubedOS as an executable file on a
  conventional operating system such as Windows. The simulated hardware logs activity for
  subsequent analysis. Unlike the unit test program, the Mock program is a complete version of
  the system as it will fly in space. None of the code in this folder needs to be SPARK.

\item \filename{CubeSat}. This folder contains the actual code needed on the CubeSat. All the
  code in this folder will fly in space and must be SPARK.

\end{itemize}

In addition to the source folders, there is a \filename{build} folder as well. This folder
contains subfolders for the various build configurations that are defined. The object files and
executable files are placed in these subfolders.

The \filename{build} folder also contains some test programs used to exercise the physical
hardware in various ways. See the \filename{README.txt} files in those subfolders for more
information.

At the time of this writing all SPARK artifacts are put into the \filename{CubeSat} build
folders. The project files thus use the CubeSat build as their default build.
