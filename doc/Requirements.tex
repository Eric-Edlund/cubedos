
\chapter{Requirements}
\label{chapt:requirements}

In this chapter we detail the requirements of the various CubedOS core modules. These
requirements are used to drive the architecture and design of the system. \pet{There are quite a
  few design-ish aspects to the description here. It might be valuable to create a better
  separation between requirements and design.}

\section{Core Requirements}
\label{sec:core-requirements}

At its heart CubedOS is a message passing microkernel. Conceptually, each component of the
system is stored in a \newterm{module} consisting of one or more Ada packages arranged in a
package hierarchy. Each module has a \newterm{module root package} that is the parent of the
module's package hierarchy. Modules in CubedOS are conceptually similar to processes in a
conventional operating system or ``applications'' in CFS.

CubedOS uses modules for application code, high level operating system-like components, and
device drivers. Applications can be constructed from multiple application level modules. The
system does not distinguish between different types of modules (for example, between application
specific modules and device drivers). There is no ``kernel mode'' of execution. The architecture
we describe here is thus enforced primarily by convention. In theory, any structuring
expressible using normal Ada visibility rules would be acceptable, but there are advantages to
following the conventions of CubedOS, as we describe throughout this document. \pet{We should
  consider building a tool that enforces the aspects of our convention that are most important.
  We should also consider writing a document that specifies all CubedOS conventions in a single,
  easy to reference place.}

For example it is possible, even if not normally desirable, for an application module to
directly invoke functionality in a device driver. In contrast, conventional operating systems
typically do not allow this, or only allow it under very controlled circumstances. However, such
direct invocations may be more efficient; an important consideration for a constrained embedded
system. This example shows that while we recommend certain design approaches when using CubedOS,
we also do not want to be overly constraining. CubedOS users with special needs, should be able
to work with the system rather than outside of it.

CubedOS does distinguish between active modules and packages of shared library code. A module is
distinguished from library code in that a module contains at least one task, whereas library
packages are entirely passive. CubedOS conventions do permit library packages to be shared among
multiple modules, but it is not permitted for any module to directly call subprograms or entries
in another module. It is also not permitted for one module to communicate with another module
via the internal state of a library package. \pet{Ada has some pragmas in the distributed
  systems annex that we can perhaps use here to provide compiler enforcement of this
  restriction.} The only way functionality in another module can be invoked is by sending that
module a message.

In each module one task selected by the programmer, called the module's \newterm{server task},
is distinguished. The server task is the \emph{only} task in a module that is allowed to receive
messages for that module. However, messages can be sent by any task. In many modules the server
task is the only task. The following requirements apply:

\begin{itemize}
\item \textbf{Core.SPARK}. The core system, including the core modules described in this
  document, shall be written in \SPARK\ and proved free of information flow errors and execution
  runtime errors.
\item \textbf{Core.Ravenscar}. The core system, including the core modules described in this
  document, shall be written to conform with the restrictions of the Ravenscar profile. \pet{We
    might want to ``upgrade'' to the Jorvik profile. This gives us: relative delays, ``pure''
    barrier expressions, and access to Ada.Calendar. It gives us other things that I think we
    are less likely to need.}
\item \textbf{Core.Static}. Modules shall be statically allocated. All modules the system will
  ever have shall exist at deployment time. There is no dynamic loading of modules. \pet{What
    about on-the-fly software updates? Are we ruling that out, or is that outside the scope of
    these requirements?}
\item \textbf{Core.UniqueID}. Each active module shall have a unique ID number that is either
  statically assigned by the application developer or dynamically assigned at system
  initialization time by the Name Service. \pet{The Name Service is controversial. We might not
    want it and instead require that module ID numbers all be statically assigned.} CubedOS core
  modules (described below) shall have fixed, ``well known'' module ID numbers.
\item \textbf{Core.IDNumbers}. Module ID numbers shall be small integers beginning at $1$ and
  ranging up to the number of modules in the system. However, if a particular core module is not
  used, it's ID number is skipped (not reassigned). \pet{We probably don't really want to waste
    module ID number space like this. Reassigning core module ID numbers for core modules that
    are not used seems like a desirable optimization that we should allow developers.}
\end{itemize}

\section{Runtime Library}
\label{sec:runtime-library}

The CubedOS runtime library is a collection of packages containing general purpose components of
interest to many CubedOS applications. Unlike the Ada standard library and many third party
libraries, the CubedOS runtime library is written in \SPARK. The following requirements apply:

\begin{itemize}
\item \textbf{Lib.SPARK}. The runtime library shall be written in \SPARK\ and proved free of
  information flow and execution runtime errors.
\item \textbf{Lib.Tests}. Unit tests shall be provided for all library components with an
  emphasis on sections that may not yet be fully proved.
\item \textbf{Lib.TaskSafety}. All library components shall be task safe in the sense that they
  can be called simultaneously from multiple tasks without error. This is needed because library
  packages are shared among the modules.
\item \textbf{Lib.PackageDeployment}. Only the packages actually used by an application shall
  become part of the application's executable. This means the library can contain a rich
  collection of packages without burdening applications that don't require them.
\end{itemize}

The precise library components provided is not specified here but is intended to be an open
ended set of components found useful in CubedOS applications. Examples of such components
include, but are not limited to:

\begin{itemize}
\item String processing, including regular expression processing
\item Statistical processing
\item Linear algebra
\item Image processing
\item Error checking and error correction
\item Communication protocol processing
\item Lightweight database support
\item Security processing such as encryption/decryption
\item Data compression
\end{itemize}

What distinguishes runtime library material from active CubedOS modules is that none of the
runtime library components contain tasks or perform blocking operations (such as delays).
Notice, however, that some library components may wish to send messages to other active
components, such as the file server. In that case the library cannot directly receive the reply
(only a module can receive from a mailbox), but instead must rely on the module using the
library to forward replies into the library as appropriate. This unusual organization also
distinguishes the CubedOS runtime library from most other libraries. \pet{Probably it would be
  better for library components to never send messages. Maybe we should make that another aspect
  of our CubedOS coding conventions.}

\section{Core Modules}
\label{sec:core-modules}

Any CubedOS based system shall contain certain \newterm{core modules} as a minimum. They always
exist and have fixed, well-known module ID numbers. Note that it is possible that some core
modules will not be required by all applications. However, their module ID numbers remain
reserved. \pet{The motivation for using well-known ID numbers is to facilitate third party
  module development.} In this section we describe these core modules.

\subsection{Message Manager}
\label{sec:message-manager}

The message manager is not a proper CubedOS module but rather forms the CubedOS kernel. It thus
does not have a module ID number. Instead the message manager holds mailboxes used to facilitate
communication between the other active modules in the system. The following requirements apply:

\begin{itemize}
\item \textbf{MessageManager.Mailbox}. The message manager shall provide exactly one mailbox for
  each module in the system.
\item \textbf{MessageManager.Access}. Mailboxes shall be accessed using the module ID number of
  the recipient. These ID numbers shall be globally available to all modules either as statically
  initialized constants or via the Name Service using abstract names as keys.
\item \textbf{MessageManager.Async}. Messages shall be delivered asynchronously to a module's
  mailbox. It is not necessary for a module to receive a sent message immediately.
\item \textbf{MessageManager.Sender}. Messages shall carry the full CubedOS address of the
  sender to so that replies can be returned.
\item \textbf{MessageManager.Unstructured}. Messages shall be unstructured octet arrays. This
  allows all modules to use a common message type. However, to simplify the task of creating and
  decoding messages, CubedOS shall provide a standard message encoding/decoding facility.
  \footnote{This facility is implemented in the Merc tool.}
\item \textbf{MessageManager.Sizes}. It shall be possible for the application developer to tune
  the size of the mailboxes (number of pending messages) and the size of the messages
  themselves.
\item \textbf{MessageManager.Priority}. Messages shall have a priority assigned to them which
  shall be, by default, the priority of the sending module.
\item \textbf{MessageManager.PrioritySet}. It shall be possible for the sender of a message to
  set the priority of that message to reduce its priority on behalf of a low priority client.
\item \textbf{MessageManager.FIFO}. Messages shall be received by a module in FIFO order by
  priority. In other words, high priority messages shall be received first (in FIFO order),
  followed by lower priority messages (also in FIFO order).
\item \textbf{MessageManager.PriorityInheritance}. When a high priority message is received, the
  server task in the receiving module shall have its priority automatically elevated to match
  that of the message it is processing. The server task's priority shall return to its previous
  setting after handling the message. \pet{This requires dynamic priorities and thus is in
    conflict with \textbf{Core.SPARK}}
\item \textbf{MessageManager.Lost}. Messages may be lost during sending. When a module attempts
  to send to a full mailbox either a) the message shall be lost with a status return to inform
  the sender, b) the message shall be lost with no status return, or c) the oldest message with
  the same or lower priority shall be purged from the mailbox. If no such message exists in the
  last case (because they all have higher priority), the message shall be lost.
\end{itemize}

The message manager only provides support for asynchronous messages. No support for synchronous
messages is needed. If a synchronous request/reply is required, an explicit reply message must
be sent. This means a module may have to field other, unrelated messages while it waits for a
reply from an earlier request. This impacts the implementation of modules since they must
remember the status of all pending requests. \pet{The CubedOS library should provide some
  assistance with this.} However, it also promotes a high degree of concurrency since, in
effect, no requests are ever blocking.

\subsection{Time Service}
\label{sec:time-service}

The Time Server module provides real-time clock services. However, due to the latency and
overhead of message passing, modules that require high speed, real-time timing services may need
to use internal tasks instead. The time server is suitable for slow speed or non-critical timing
services. \pet{This needs to be investigated further. What can be specifically said about the
  timing of the message passing service?} The following requirements apply:

\begin{itemize}
\item \textbf{Tick.RealTime}. The time server shall use a real time clock as the basis of its
  timing. Such a clock is monotonic and tracks time independently of any software activity.
\item \textbf{Tick.Periodic}. The time server shall provide a service whereby a module can
  request the delivery of periodic \newterm{tick messages} with a period ranging from 1 ms to 1
  hour (3,600,000 ms). The total set of tick messages generated in response to such a request is
  called a \newterm{periodic series}.
\item \textbf{Tick.Latency}. The time between when a request for a periodic series is received
  by the time server and when the first tick message in that series is sent shall be not more
  than one millisecond plus the latency due to message passing and processing.
\item \textbf{Tick.OneShot}. The time server shall provide a service whereby a module can
  request the delivery of a single tick message either after a specified delay (in the range
  from 1 ms to 1 hour), or at a specified absolute time. In this case the response message
  constitutes a \newterm{one shot series} of size one.
\item \textbf{Tick.SeriesID}. Every request shall contain a \newterm{series ID}, specified by
  the requester, that identifies a particular series.
\item \textbf{Tick.MultiSeries}. The time server shall support multiple series being
  delivered to the same module. For example a periodic series and a one shot series, or multiple
  periodic series with different periods, all going to the same module is supported.
\item \textbf{Tick.Message}. Tick messages shall contain a counter, starting at one, that
  indicates which message in the series it is.\footnote{The counter shall always be one in the
    case of one shot ticks.} Also tick messages shall contain the series ID number.
\item \textbf{Tick.Cancel}. The time server shall accept messages that cancel a
  pending\footnote{For example a one shot tick that has not yet occurred can still be canceled}
  or active series. The cancellation message shall contain the series ID. If the ID is invalid,
  there shall be no effect. \pet{Implementing cancellation could be tricky and it may not be
    important. This requirement should be considered low priority.}
\end{itemize}

\subsection{File Service}
\label{sec:file-service}

Not all missions may require a file system, but many will. The core File Server module provides
a simple file system interface, but the precise implementation is not specified here. It could
be implemented as a layer around a host operating system API, through a third party library, or
even as a RAM disk. The following requirements apply:

\begin{itemize}
\item \textbf{FileServer.Names}. File names shall be limited to the FAT16 style ``8.3'' naming
  convention. This is a least common denominator naming convention that should be supportable by
  almost any underlying implementation. \pet{What should we say about the case sensitivity of
    file names?}
\item \textbf{FileServer.Size}. File sizes shall be represented using an unsigned integer with
  at least 31 bits. This supports files potentially as large as $2^{31} - 1$ bytes in size,
  depending on the underlying implementation.
\item \textbf{FileServer.FileOps}. The following operations shall be allowed on files: Open,
  Close, Read, and Write.
\item \textbf{FileServer.MultiOpen}. It shall be possible to have multiple files open at once.
  The limit on the number of simultaneously open files is not specified here but should be
  ``reasonable.''
\item \textbf{FileServer.GlobalHandles}. File handles (representing open files) shall be global
  for the entire system. In particular a handle sent from one module to another shall be usable
  by the receiving module to access a file opened by the sending module.
\item \textbf{FileServer.Modes}. It shall be possible to open files for reading, writing, and
  append mode.
\item \textbf{FileServer.Access}. It shall be possible to open files for either sequential or
  random access.
\item \textbf{FileServer.Sharing}. A file shall be openable many times simultaneously for
  reading, but any writing or appending shall require exclusive access.
\item \textbf{FileServer.Binary}. There shall be no distinction between text files and binary
  files. All data transfers shall be explicit (if a carriage return or line feed character is
  desired, it shall be written explicitly and processed explicitly on reads).
\item \textbf{FileServer.Directory}. A single directory of files shall be provided.
  Subdirectories shall not be provided.
\item \textbf{FileServer.Attributes}. In addition to its name, a file shall also have the
  following attributes: Size, Date/Time stamp when last modified. The Date/time stamp shall be
  updated whenever the file is created or written.
\item \textbf{FileServer.DirOps}. The following operations shall be allowed on directories:
  EnumerateFiles, DeleteFile, GetFileAttributes.
\end{itemize}

Roughly speaking the file server interface should provide all the usual file operations, but
with a minimum of extra functionality. The maximum size of data transfered when reading or
writing files is left unspecified. It is recommended that implementations allow transfers up to
maximally sized messages.

\pet{Do we want some kind of ``garbage collection'' of open file handles? Maybe have handles
  automatically closed if they are inactive too long?}

\subsection{Interpreter Service}
\label{sec:interpreter-service}

The Interpreter module shall be able to run mission specific commands at a set time, expressed
either as an absolute time or as a time interval relative to the previously issued command. It
shall make use of the Time Server, so all limitations of that service also apply here. The
Interpreter is meant to issue raw CubedOS messages to other modules. At this time it is not
required to have any inherent command language of its own. However, the addition of ``meta
commands'' that support control flow or subroutines might be an appropriate avenue for future
work. The following requirements apply:

\begin{itemize}
\item \textbf{Interpreter.Commands}. The commands executed by the Interpreter shall be CubedOS
  messages consisting of a target address triple (Domain ID, Module ID, Message ID) and message
  arguments already encoded into an octet stream.
\item \textbf{Interpreter.Ignore}. The Interpreter shall ignore all reply messages sent to it in
  response to the command messages it sends. The only incoming messages the Interpreter will
  process shall be those specifically directed to control its functionality (i.e., to provide it
  with the command sequence to later interpret).
\item \textbf{Interpreter.Scheduler}. Each command given to the Interpreter shall have an
  associated time stamp, either an absolute time, a time relative to the previous command, or a
  time relative to \newterm{deployment time}\footnote{The absolute time when the system is
    deploy after launch.}. The Interpreter shall issue commands at their designated time.
  \pet{What should happen if the Interpreter encounters a command that must be issued in the
    past? Should it ignore the command or issue it ``immediately?''}
\item \textbf{Interpreter.ScriptBuilder}. A separate tool shall be provided that can be used to
  prepare the input for the Interpreter. It as assumed that hand preparation of that input will
  be excessively difficult and error prone.
\end{itemize}

The ScriptBuilder tool has its own requirements, design, and implementation, and is decribed
elsewhere.

\subsection{Log Service}
\label{sec:log-service}

The Log Server module is used to log messages from other modules. \pet{Library components
  shouldn't attempt to log messages directly. They should return appropriate status indications
  to the calling module and that module can then log an appropriate message if necessary.} Here
we use ``log'' expansively to include telemetry messages. The log can be stored on non-volatile
storage, printed on a console (if one is available), or gathered in a data structure for later
transmission to a ground station. Unlike telemetry ``data'' which might be in an unreadible
binary form, telemetry messages are expected to be human readible. The following requirements
apply:

\begin{itemize}
  \item \textbf{Log.MessageSize}. Log messages are intended to be relatively short and should be
    limited to only 128 characters at most. This effectively creates a lower bound on the size
    of CubedOS messages at 128 octets since log messages are transmitted to the Log Server via
    the usual CubedOS message system. Restricting log messages to 128 characters, however, means
    that all useful instantiates of the generic Message Manager will be able to pass log
    messages without issue. The Log Server will not impose a large message size on applications
    that don't need it.
  \item \textbf{Log.CharacterSet}. Log messages shall use only printable ASCII characters, free
    from all control characters (including tabs, carriage return, or line feed). This will
    minimize problems with processing the messages later.
  \item \textbf{Log.Language}. Log messages shall be written in English using, as much as
    possible, standard (American) English spelling and punctuation. Log messages are intended to
    be human readible.
  \item \textbf{Log.Level}. The Log Server shall allow messages to be marked with different
    ``levels'' including (but not limited to): debug, informational, warning, error, and
    critical levels. This can be used to classify each message with respect to its criticality.
  \item \textbf{Log.Source}. The Log Server shall automatically prefix log messages with an
    indication of which module sent the message (domain ID and module ID).
  \item \textbf{Log.Timestamp}. The Log Server shall automatically record a timestamp for each
    log message that it receives and prefix the message with the timestamp. The timestamp shall
    have the format: YYYY-MM-DD HH:MM:SS.SSS based on the system clock running on the
    spacecraft. It is outside the scope of the Log Server's requirements to define how that
    clock is set and how its time is maintained. Those details are likely to be
    mission-specific.
  \item \textbf{Log.Storage}. The Log Server shall store a certain number of log messages and
    make them available for later transmission as telemetry messages. \pet{How many messages
      should it store? Should that be configurable?}
\end{itemize}

\section{Low Level Abstraction Layer}
\label{sec:llal}

\pet{Need to talk about the modules that interface to standard busses and other low level
  hardware resources. The idea is that ``high level'' device drivers could be written in terms
  of low level operations making it possible to port a CubedOS application to a new platform by
  only rewriting the LLAL modules.}
