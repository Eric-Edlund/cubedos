%%%%%%%%%%%%%%%%%%%%%%%%%%%%%%%%%%%%%%%%%%%%%%%%%%%%%%%%%%%%%%%%%%%%%%%%%%%%
% FILE    : main.tex
% AUTHOR  : (C) Copyright 2018 by Sean Klink & Carl Brandon & Peter Chapin
% SUBJECT : Paper on CubedOS
%
% TODO: Write me! Add more specific items here as they become apparent.
%
% Send comments or bug reports to:
%
%    Carl Brandon
%    Vermont Technical College
%    Randolph Center, VT 05061
%    CBrandon@vtc.vsc.edu
%%%%%%%%%%%%%%%%%%%%%%%%%%%%%%%%%%%%%%%%%%%%%%%%%%%%%%%%%%%%%%%%%%%%%%%%%%%%

%+++++++++++++++++++++++++++++++++
% Preamble and global declarations
%+++++++++++++++++++++++++++++++++
\documentclass{llncs}
\usepackage{listings}
\usepackage{gensymb}
\usepackage{graphicx}
%\usepackage{unicode-math}
\usepackage{url}
\usepackage{hyperref}


\newcommand{\newterm}[1]{\emph{#1}}    % For newly introduced terms.
\newcommand{\code}[1]{\texttt{#1}}     % Used for bits of Ada inline.
\newcommand{\filename}[1]{\texttt{#1}} % For file names.
\newcommand{\SPARK}{\textsc{Spark}}    % For ease of small capping SPARK


% Redefine the underscore command so latex will break words at underscores. Found this trick at
% http://mrtextminer.wordpress.com/2009/02/26/break-a-long-word-containing-underscores-in-latex/
%
\let\underscore\_
\newcommand{\breakingunderscore}{\renewcommand{\_}{\underscore\hspace{0pt}}}

% Issue the changed underscore command to the whole document.
\breakingunderscore


% The \note macros are useful for creating easy to see notes.

% Carl Brandon
\long\def\car#1{\marginpar{CAR}{\small \ \ $\langle\langle\langle$\
{#1}\
    $\rangle\rangle\rangle$\ \ }} 

% Peter Chapin
\long\def\pet#1{\marginpar{PET}{\small \ \ $\langle\langle\langle$\
{#1}\
    $\rangle\rangle\rangle$\ \ }} 

% Niels Huisman
\long\def\nie#1{\marginpar{NIE}{\small \ \ $\langle\langle\langle$\
{#1}\
    $\rangle\rangle\rangle$\ \ }} 

% Nicole Hurley
\long\def\nic#1{\marginpar{NIC}{\small \ \ $\langle\langle\langle$\
{#1}\
    $\rangle\rangle\rangle$\ \ }} 


%% The following are settings for the listings package. See the listings package documentation
%% for more information.
%%
%\lstset{language=C,
%        basicstyle=\small,
%        stringstyle=\ttfamily,
%        commentstyle=\ttfamily,
%        xleftmargin=0.25in,
%        showstringspaces=false}

% Define listing parameters for Ada 2012. Base it on Ada 2005.
\lstdefinelanguage{Ada2012}[2005]{Ada}
{
  morekeywords={some},
  sensitive=false,
  breaklines=false,
  showstringspaces=false,
  xleftmargin=0.25in,
  basicstyle=\small\sffamily,
  columns=flexible,
}
\lstset{language=Ada2012, showlines=true}

% Wide equivalent of the listings package lstset. The parameter is the amount to indent the code
% (may be negative).
%
\lstnewenvironment{widelisting}[1]
   {\lstset{language=Ada2012,xleftmargin=#1}}
   {}
   
% Wide listing from a file using the Listings package lstinputlisting. First parameter is
% path name to file (can be relative). Second parameter is the amount to indent the code (may
% be negative).
%
\newcommand{\wideinputlisting}[2]{%
   \lstinputlisting[language=Ada2012,xleftmargin=#2]{#1}}


% An environment for displaying use cases.
%   This environment takes three parameters:
%   \param #1: The name of the use case.
%   \param #2: The actor who participates in the use case.
%   \param #3: The context in which the use case executes.
%
%   The body of the environment is the action associated with the use case.
\newsavebox{\UseCaseName}    % Create some boxes to hold the necessary text.
\newsavebox{\UseCaseActor}   % We need to do this because we can't use the
\newsavebox{\UseCaseContext} % environment parameters in the 'end' definition.
\newsavebox{\UseCaseAction}
\newenvironment{usecase}[3]
 {
  \sbox{\UseCaseName}{\bfseries #1}  % The name is easy.
  \sbox{\UseCaseActor}{#2}           % The actor is easy.
  \begin{lrbox}{\UseCaseContext}     % Format the context in a minipage.
    \begin{minipage}{3.25in}
    #3
    \end{minipage}
  \end{lrbox}
  \begin{lrbox}{\UseCaseAction}      % The environment body becomes the action.
  \begin{minipage}{3.25in}}
 {\end{minipage}
  \end{lrbox}
\begin{center}   % Now spew forth the table using the information collected above.
\begin{tabular}{|l||p{3.5in}|} \hline
\multicolumn{2}{|c|}{\usebox{\UseCaseName}} \\ \hline
Actor   & \usebox{\UseCaseActor}   \\ \hline
Context & \usebox{\UseCaseContext} \\ \hline
Action  & \usebox{\UseCaseAction}  \\ \hline
\end{tabular}
\end{center}}


%++++++++++++++++++++
% The document itself
%++++++++++++++++++++
\begin{document}

\title{Message Encoding for CubedOS Using the XDR2OS3 Code Generator}
\author{Sean Klink\inst{1} \and Peter Chapin\inst{3} \and Carl Brandon\inst{2} }

\institute{Vermont Technical College, Randolph Center VT 05061, USA,\\
  \email{sean.klink@vtc.edu} %,\\
      % WWW home page: \texttt{http://web.vtc.edu/users/xyzzy}
  \and
  \email{carl.brandon@vtc.edu} %,\\
      % WWW home page: \texttt{http://web.vtc.edu/users/xyzzy}
  \and
  Vermont Technical College, Williston VT 05495, USA,\\
  \email{pchapin@vtc.vsc.edu} %,\\
      % WWW home page: \texttt{http://web.vtc.edu/users/pcc09070}
}

\maketitle

\begin{abstract}
  % Summarize the contents of the paper using no more than 400 words.

  We are developing the flight software the Lunar IceCube mission, scheduled to be launched in
  2018. Lunar IceCube will be a 6-U CubeSat orbiting the moon in search of water, ice, and other
  volatiles near the lunar surface. Lunar IceCube will be the first spacecraft to use CubedOS, a
  general purpose, open source application framework for CubeSat flight software written by us.
  CubedOS is written in \SPARK\ 2014 and proved free of certain classes of runtime error. It
  uses a message passing architecture where each concurrently executing module processes
  messages from an associated mail box and sends messages to other mail boxes. CubedOS
  encourages a client/server software architecture where server modules process request messages
  and return reply messages to clients. Message data is encoded using the external data
  representation (XDR) standard. Each module thus provides encoding and decoding subprograms so
  the clients of a module are not required to manage message encodings themselves. These
  subprograms are tedious for module authors to produce manually. In this paper we present
  XDR2OS3, a tool that enables CubedOS module authors to specify message structures using an
  interface language that follows the XDR standard, extended to account for the special
  properties of \SPARK\ (such as constrained subtypes). XDR2OS3 uses the message structures and
  generates a \SPARK\ package containing suitable encoding and decoding subprograms. The output
  of XDR2OS3 can be analyzed by the \SPARK\ tools and proved free of runtime error. While the
  development of XDR2OS3 is a work in progress, our preliminary experience with the tool is
  encouraging. It allows module authors to focus on message types and how they will be used and
  supported rather than on the low level details of message formatting. We also discuss the
  potential for XDR2OS3 to be used in other \SPARK\ applications with an interest in XDR encoded
  data.
  
  \keywords{SPARK, student project, CubeSat}
\end{abstract}


\section{Introduction}

CubeSats are small (10\,cm x 10\,cm x 10\,cm) spacecraft that are in a size class often referred
to as nano-satellites. Typical CubeSat missions are executed in low Earth orbit. They are placed
there with the help of much larger launch vehicles that are launching full sized satellites for
some unrelated purpose (e.g., satellite television). In recent years interest is rising in using
CubeSats on more ambitious missions such as accompanying deep space spacecraft on missions to the
moon and planets. Deep space CubeSat-only missions are also being contemplated. \pnote{We need
  some citations here!}

As with all spacecraft, software plays a critical role in the success of CubeSat missions.
Unfortunately many CubeSat missions have been plagued with catastrophic software problems
\pnote{More citations, please}. Historically CubeSats have been created in a relatively informal
fashion by organizations, such as universities, with very limited budgets and inexperienced
personnel, such as students. Commonly the emphasis of the development effort is on the physical
and electrical engineering of the spacecraft with software development given only incidental
attention. The result is often inadequate, faulty software.

As CubeSat missions grow even more ambitious, software complexity and the requirements for
software reliability grow accordingly. Because of limited physical space and other resources, as
well as limited development and support personnel, CubeSat missions often must dispense with the
levels of redundancy and review afforded to high profile, multi-billion dollar NASA/ESA
missions. This further exacerbates the problems with creating robust software for CubeSats.

At Vermont Technical College's CubeSat Laboratory we are working to address these issues. We are
developing a general purpose, open source application framework for CubeSat flight software that
we call \newterm{CubedOS}. While not precisely an operating system (it is intended to run on top
of an existing operating system or runtime environment), CubedOS does provide some basic
infrastructure and supporting services of interest to flight software developers. A well
documented reusable system such as CubedOS will simplify the construction of flight software and
improve its reliability by cutting down on the amount of supporting code that must be developed
by each mission.

NASA's Goddard Spacecraft Flight Center has, in fact, developed a very similar framework called
the Core Flight System (CFS) \pnote{Citation needed}. Unlike current CubedOS, CFS is well
established with a large community of users and several active and planned missions
\pnote{Citation needed}. In fact, it is our goal to eventually provide a CubedOS/CFS bridge that
would allow a current CFS-based mission to migrate to CubedOS incrementally if desired.

However, unlike CFS, which is written in unverified C, CubedOS is entirely written in \SPARK\
2014 and verified free of runtime error in the sense meant by \SPARK. We also intend to use
\SPARK, as time and resources allow, to prove that CubedOS satisfies certain functional
correctness properties as well.

CubedOS applications are organized as a collection of \newterm{modules}. Each module is usually
implemented as a hierarchy of Ada packages with a single root. Every module has an ID number
assigned to that module by the application developer. System modules providing basic services
useful to many, if not most, missions are given ``well known'' ID numbers by a central authority
following the approach used when assigning well known port numbers for TCP based network
services.

Modules communicate with each other via a fairly standard, asynchronous message passing system.
Each module is associated with exactly one \newterm{mailbox} from which it receives all messages
destined to it. Modules send messages by adding the messages to the mailboxes of the recipients.
The mailboxes are implemented as protected objects.

Every module contains at least one task that reads that module's mailbox and processes the
messages received. Modules may contain additional, internal tasks if necessary. Because \SPARK\
limits the number of entries on a protected object to one, the |Send| procedure is implemented
as a protected procedure and not an entry. One consequence of this design is that messages might
be lost of a mailbox fills to capacity. There is a way for a message sender to detect such loss
if desired.

The mailboxes in CubedOS are organized in \newterm{communication domains} consisting of an array
of mailboxes indexed by module ID numbers. CubedOS supports the use of multiple communication
domains identified by domain IDs set by the application developers. Module IDs are scoped by
domain IDs so, for example, two different modules can have the same Module ID as long as they
are in different domains. This architecture anticipates the extension of CubedOS to distributed
systems, such as CubeSat ``swarms'' as envisioned by some mission planners.

Because CubedOS mailboxes are stored in an array, they must all have the same data type.
However, this is problematic because the messages received by the various modules will, in
general, contain different collections of typed parameters. Thus the messages are stored in the
mailboxes in a raw binary format as strings of 8~bit octets. \pnote{Do we need to say something
  about how the request/reply format requires this architecture?} This means the typed message
arguments must be encoded in some way into raw binary data before sending and then decoded by
the receiver back into properly typed message parameters.

This encoding and decoding process closely resembles the marshaling and unmarshaling of
procedure arguments that takes place in the context of a remote procedure call (RPC) protocol.
Although CubedOS has been conceptualized as a message passing system, it could just as easily be
thought of as providing an asynchronous RPC discipline between modules.

Following the tradition of many RPC systems \pnote{Citation needed}, CubedOS allows the
developer to describe the interface of a module in specialized interface definition language. A
tool is then provided, the main subject of this paper, that compiles the interface definition
into packages to perform the necessary encoding and decoding of message parameters. Since we
selected the External Data Representation (XDR) standard for the binary format, we call our tool
XDR2OS3. The interface definition language we use is an extension of that described in \pnote{add
  citation to the RFC} which we call \newterm{modified XDR} (MXDR).

XDR2OS3 outputs \SPARK\ packages that can be proved free of runtime error. The modifications in
the MXDR language account for the special needs of the \SPARK\ target language. In particular:

\begin{itemize}
\item We extended XDR to allow the declaration of constrained ranges on types. These constraints
  are mapped to \SPARK\ declarations in an obvious way. This was necessary since the use of
  constraints plays an important role in making the code provable.

\item We defined a special kind of structure called a \newterm{message struct} that informs
  XDR2OS3 when to generate encoding and decoding subprograms. The members of a message struct
  correspond to the parameters of a message. Ordinary structures are mapped to \SPARK\ records
  and, by themselves, do not define messages.

\item We allow a message struct to optionally carry a \newterm{message invariant} that describes
  a condition that must hold on the values inside the message struct. That condition is mapped
  to a precondition on the encoder subprogram (to ensure the parameters of the encoder have the
  invariant), and to a postcondition on the decoder subprogram (to inform the calling
  environment about the invariant).
\end{itemize}

The rest of this paper is organized as follows. Section~2 gives more details about the MXDR
language. Section~3 describes the code generated by XDR2OS3. Section~4 discusses the performance
of the generated code and related issues. Section~5 concludes. We make use of a running example
throughout.


\section{Conclusion}

XDR2OS3 is wonderful. Using it will change your life forever.


\bibliographystyle{splncs03}
\bibliography{../../LaTeX/references-General,../../LaTeX/references-SPARK,../../LaTeX/references-CubeSat}

\end{document}
