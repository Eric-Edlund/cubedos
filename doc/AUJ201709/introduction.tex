
\section{Introduction}

CubedOS is being developed at Vermont Technical College's CubeSat Laboratory with the purpose of
providing a robust software platform for CubeSat missions and of easing the development of
CubeSat flight software. In many respects the goals of CubedOS are similar to those of the Core
Flight Executive (cFE) written by NASA Goddard Space Flight Center \cite{cFE}. However, unlike
cFE, CubedOS is written in \SPARK\ and verified to be free of the possibility of runtime error.
\SPARK\ has also been used to provide some other correctness properties in certain cases. We
compare CubedOS and cFE in more detail in Section~\ref{section-related-work}.

The intent is for CubedOS to be general enough and modular enough for many groups to profitably
employ the system. Since every mission uses different hardware and has different software
requirements, CubedOS is designed as a framework into which \newterm{modules} can be plugged to
implement whatever mission functionality is required. CubedOS provides inter-module
communication and other common services needed by many missions. CubedOS thus serves both as a
kind of operating environment and as a library of useful tools and functions.

Some of the module functionality useful for complex CubeSat missions would include interfaces to
 attitude determination and control systems (ADACS), electrical power systems (EPS), photovoltaic
 panel orientation gimbals, navigation and data radio, data collection instruments, thermal control
 radiators, ion engine with gimbals, and cameras. We also plan on including a specific module for
 spiral thrusting which allows for three axis angular momentum control with a two axis thruster.

It is our intention that all CubedOS modules also be written in \SPARK\ and at least proved free
of runtime error. However, CubedOS allows modules, or parts of modules, to be written in full
Ada or C. This allows CubedOS to take advantage of third party C libraries or to integrate with
an existing C code base.

CubedOS runs on top of the Ada runtime system and thus works with any underlying platform
supported by the available Ada compiler. For example, CubedOS makes use of Ada tasking without
directly invoking the underlying system's support for threads. This simplifies the
implementation of CubedOS while improving its portability. However, CubedOS does require that a
rich Ada runtime system be available for all envisioned targets. Specifically, CubedOS requires
a runtime system that supports the Ravenscar profile.

For resources that are not accessible through the Ada runtime system, CubedOS driver modules can
be written that interact with the underlying operating system or hardware more directly.
Although these modules would not be widely portable, they could, in some cases, be written to
provide a kind of low level abstraction layer (LLAL) with a portable interface. We have not yet
attempted to standardize the LLAL interface. However, we see that as an area for future work.

CubedOS applications are organized as a collection of active and passive modules, where each
active module contains one or more Ada tasks. Passive modules do not contain any tasks but are
used as containers for shared, reusable code. Although CubedOS is written in \SPARK\ there need
not be a one-to-one correspondence between CubedOS modules and \SPARK\ packages. In fact,
modules are routinely written as a collection of Ada packages in a package hierarchy.

Critical to the plug-and-play nature of CubedOS, each active module is self-contained and does
not make direct use of any code in any other active module, although passive modules serving as
library components can be used. All inter-module communication is done through the CubedOS
infrastructure with no direct sharing of data or executable content. In this respect CubedOS
active modules are similar to operating system processes. One consequence of this policy is that
a library used by several modules must be either duplicated in each module, for example as
private child packages, or provided as an independent, passive module. In this respect passive
modules are similar to operating system shared libraries and have similar concerns regarding
task safety and global data management.

In the language of operating systems, CubedOS can be said to have a microkernel architecture
where task and memory management is provided by the Ada runtime system. Both low level
facilities, such as device drivers, and high level facilities, such as communication protocol
handlers, are all implemented as CubedOS modules. All modules are treated equally by CubedOS;
any layered structuring of the modules is imposed by programmer convention.

% TODO: Some of this discussion, along with some diagrams, should perhaps be moved to the
% section describing the architecture of CubedOS. Some orientation to the system is appropriate
% in the introduction... but how much?

CubedOS is currently a work in progress It is our intention to release CubedOS as open source
once it is more mature and refined. We also need to review the code base to verify that it is
free from International Traffic in Arms Regulations (ITAR) restrictions and possibly release both
 ITAR compliant and U.S non ITAR compliant versions. We anticipate this to happen in mid-2018.
