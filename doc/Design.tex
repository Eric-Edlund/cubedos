<
\chapter{Design}
\label{chapt:design}

In this section we describe the design of CubedOS, including various design trade-offs made.
This information is primarily of interest to developers looking to enhance the CubedOS
infrastructure itself. Developers interested in only using CubedOS do not need to read this
section unless they are interested in the rationale behind some of CubedOS's design decisions.

\section{Message Manager}
\label{sec:design-message-manager}

One of the central problems facing the message manager is the handling of message mailboxes.
Since \SPARK\ does not allow dynamically allocated memory, any method that involves dynamically
allocating space for messages is not an option. Instead mailbox space must be allocated
statically. This forces users of CubedOS to anticipate the maximum sizes of the mailboxes they
require.

We note that an upcoming version of \SPARK\ \emph{will} allow the use of dynamically allocated
memory, although in a restricted way. This opens the possibility of revisiting the design
described here and perhaps working around some of the issues we identify below. We have not yet
investigated this in detail; \SPARK's support for dynamically allocated memory is still
immature.

We considered three strategies for managing mailboxes.

\begin{itemize}
\item \emph{Define a Mailbox type that can hold a fixed number of pending messages. Create a
    separate mailbox for each module.} This approach has the advantage of being simple. It also
  separates the total mailbox space by module. The value of the separation is that should one
  mailbox fill due to a dead or slow module, that won't impact the communication between other
  modules using different mailboxes.

  The main problem with this approach is that it tends to waste space. We assume most modules
  won't require a large number of pending messages (a large mailbox size). Yet if even one
  module does require a large size, the |Mailbox| type must be adjusted to accommodate it,
  causing unnecessarily large mailboxes in most cases.

\item \emph{Define a single Mailbox object that holds pending messages for all modules
    together.} This approach addresses the disadvantage of the previous approach by
  consolidating the free space into a single mailbox object. Modules requiring a large number of
  pending messages can in effect ``borrow'' free mailbox space from modules with few pending
  messages.

  Implementing this design is more complicated, but a bigger disadvantage is that modules can
  interfere with each other. If one module is slow and gets backlogged, the (one and only)
  mailbox could fill up with messages for that module, thus preventing communication between
  other, unrelated modules. While some workaround to this problem could probably be implemented,
  we feel this is a serious enough problem to rule out this design choice.

\item \emph{Define a Mailbox type that is discriminated on its size. Create a separate mailbox
    for each module, tuned for that module's needs.} This is a variation of the first approach
  except here the mailbox instances can each have a different size. The author of a module
  defines the size of the mailbox needed for that module; large mailboxes are only used where
  necessary and space is conserved. This approach also maintains the freedom from interference
  provided by the first approach.
\end{itemize}

The third approach is appealing but unfortunately does not allow replies to arbitrary modules in
an environment where pointers are prohibited. Although a target mailbox can be mentioned by name
when sending a message, the destination of any reply must be computed at runtime. Without
pointers there is no way to create a map from module ID number to the mailbox for that module.
The first approach can work around this problem by putting all mailboxes in a common array and
using the module ID number as an array index. However, that also requires all mailboxes to be
the same size.

A more subtle problem with the third approach is that, in general, the author of a module can't
know the maximum number of pending messages that are appropriate to specify for that module. The
correct number is a system-wide artifact that depends in large measure on the number and rate of
message send operations done by other modules. Setting the mailbox sizes is something best done
during system integration, and that is more easily accomplished if all mailboxes are in their
own package.

For these reasons CubedOS follows the first approach, using a generic package to supply the
mailboxes. The package can be instantiated differently for each CubedOS application with mailbox
sizes set appropriately for that application.

Selecting the most appropriate size for a the mailboxes is the one major issue remaining. If a
mailbox fills during operation, sending further messages to that mailbox will fail. Senders do
not block because doing so would introduce the risk of deadlock. Thus failure to send a message
due to a full mailbox creates awkwardness for senders who must manually retry the send
operation, if desired. Furthermore senders will not likely want to retry sending indefinitely
and risk being tied up doing a send operation that will never succeed. This leaves open the
question of how the failed send should ultimately be handled.

It would be better if mailbox sizes were tuned so that sending could never fail. However doing
this requires high level reasoning about the flow of messages in the system. We propose
constructing a formal model of module communication that includes finite sized mailboxes, and
then use a model checker (such as \texttt{spin}?), or some other tool, to find a configuration
of mailbox sizes where no send operation can ever fail. Doing this is an avenue for future work.

In lieu of such an analysis, human reasoning could be used to estimate appropriate mailbox sizes
followed by rigorous testing. Of course, such an approach is likely to be more error prone than
the formal approach.

If it can be formally shown that no send operation can ever fail, all calls to the |Send|
procedure could be replaced with calls to |Unchecked_Send| which does not return an error
indication. Then all error handling in the modules related to failed sending could be removed.
This would be a great simplification. However, we do \emph{not} recommend this approach. Despite
any formal modeling we might do in CubedOS's initial design, we foresee several possible ways
mailboxes might overflow anyway:

\begin{enumerate}
\item Our formal model is incorrect.
\item We never get around to creating a formal model that works.
\item Users of CubedOS might not be in a position to augment the formal model to account for
  their own, application-specific modules.
\item A module thread dies for reasons outside the software's control (such as hardware
  failure). Although it might be possible to incorporate this scenario into the formal model,
  forcing freedom from mailbox overflow in this situation might impose unreasonable requirements
  on mailbox size or application architecture (or both).
\end{enumerate}

For these reasons we advocate continued checking for failed send operations even if none are
expected in normal operation.

\pet{We should also say a few words about selecting the maximum size of messages, and about the
  handling of module ID numbers. For example, we originally considered allowing ID numbers to be
  dynamically assigned and then using a name resolver module to map module names to their
  (dynamic) ID numbers. However, that design was overly complicated.}

\section{Message Encoding}
\label{design-message-encoding}

To remain generic the messages sent between modules are essentially nothing more than arrays of
raw octets. Any structure on the information in a message is imposed by agreement between the
sender and receiver of that message. This creates a degree of type unsafety and requires modules
to defend against malformed messages at runtime. To mitigate this problem, each module contains
an API package that provides convenience subprograms that encode and decode messages. The
parameters of these subprograms are strongly typed, of course, so as long as they are used
consistently compile-time type checking is retained.

Since the problem of encoding and decoding typed data has been solved before we leverage an
existing standard, name External Data Representation (XDR) \cite{rfc-4506}. This standard
defines a ``wire format'' (in our case a message data format) for various types and simple data
structures. The advantage of using XDR is that it is well specified and understood. Using it
simplifies our specification needs and invites the use of third party tools to manage message
data.

We also considered using Abstract Syntax Notation One (ASN.1) \cite{asn.1} as a message data
format. Unlike XDR, ASN.1 includes type information in the data stream allowing for more
flexible message structures. However, ASN.1 also has more overhead, both in terms of space used
in the message and time used to encode and decode the messages. For this reason we elected to
use the simpler XDR approach.

The encoding/decoding subprograms in the API packages must contend with the richness of Ada's
type system. In particular they should be able to handle multiple parameters of various types,
called \newterm{messageable types}, where a messageable type is defined inductively as follows:

\begin{itemize}
\item An integer type, an enumeration type, or a floating point type.
\item An array of messageable types.
\item A record containing only messageable type components.
\end{itemize}

Roughly a messageable type is any type allowed by the XDR standard (although we do not aim to
provide support for XDR discriminated union types at this time).

The encoding and decoding of messageable types is tedious and repetative. Supporting subprograms
to handle basic scalars and simple arrays are provided by way of a library package
|CubedOS.Lib.XDR|. The handling of XDR arrays and records can be done using the lower level
subprograms provided by the library.

Since the creation of the API packages is largely mechanical, CubedOS comes with a tool,
\texttt{Merc}, that converts an textual representation of the message types into a corresponding
API package written in provable \SPARK. In effect, \texttt{Merc} plays a role similar to that
played by \texttt{rpcgen} in many ONC RPC \cite{rfc-5531} implementations. In particular, it
converts a high level interface description into code that encodes/decodes messages to/from the
implementer of that interface. However \texttt{Merc} differs from other similar tools in several
important respects:

\begin{itemize}
\item It targets CubedOS; the generated code makes use of CubedOS's XDR library for handling
  primitive types.
\item It generates provable \SPARK\ 2014 code.
\item It contains extensions to the language in \cite{rfc-4506} that support Ada's ability to
  define range constrained scalar subtypes.
\end{itemize}

A CubedOS application developer defines the messages she needs for a module in an XDR file and
then uses \texttt{Merc} to generate the API package and related materials. The developer can
then the subprograms in that package to encode messages for that module or decode messages from
that module. In addition, \texttt{Merc} provides subprograms the module author can use to decode
incoming messages and encode replies.

Currently it is still possible for a develper to directly access message data but doing this is
strongly not recommended. \pet{This could possibly be enforced by making the message a private
  type. However, that would require all the generated API packages to be children of the generic
  message manager package (and hence generic themselves). This might be an overly complicated
  design.}
