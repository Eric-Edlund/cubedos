\chapter{Introduction}
\label{chapt:introduction}

This document describes the design, implementation, and usage of the CubedOS operating system.
CubedOS was developed at Vermont Technical College's CubeSat Laboratory with the purpose of
providing a robust software platform for CubeSat missions and of easing the development of
CubeSat flight software. In many respects the goals of CubedOS are similar to those of the Core
Flight Executive (cFE) written by NASA Goddard Space Flight Center \cite{cFE}. However, unlike
cFE, CubedOS is written in \SPARK\ with critical sections verified to be free of the possibility
of runtime error. \SPARK\ has also been used to provide some other correctness properties in
certain cases. We compare CubedOS and cFE in more detail in Section~\ref{chapt:related-work}.

Although CubedOS was developed to support the CubeSat missions at Vermont Technical College, the
intent is for CubedOS to be general enough and modular enough for other groups to profitably
employ the system. Since every mission uses different hardware and has different software
requirements, CubedOS is designed as a framework into which \newterm{modules} can be plugged to
implement whatever mission functionality is required. CubedOS provides inter-module
communication and other common services required by many missions. CubedOS thus serves both as a
kind of operating environment and as a library of useful tools and functions.

It is our intention that all CubedOS modules also be written in \SPARK\ and at least proved free
of runtime error. However, CubedOS also allows modules, or parts of modules, to be written in
full Ada or C if appropriate. This allows CubedOS to take advantage of third party C libraries
or to integrate with an existing C code base. It is a work in progress to create a bridge
between CubedOS and cFE so that current cFE/CFS users can migriate mission critical components
to CubedOS incrementally.

CubedOS can run directly on top of the hardware, or on top of an operating system such as Linux
or VxWorks, with the assistance of the Ada runtime system. In particular, CubedOS makes use of
Ada tasking without directly invoking the underlying system's support for threads. This
simplifies the implementation of CubedOS while improving its portability. However, it does
require that an Ada runtime system be available for all envisioned targets.

For resources that are not accessible through the Ada runtime system, CubedOS driver modules can
be written that interact with the underlying operating system or hardware more directly.
Although these modules would not be widely portable, they could, in some cases, be written to
provide a kind of low level abstraction layer (LLAL) with a portable interface. We have not yet
attempted to standardize the LLAL interface, although this would be an area for future work.

The architecture of CubedOS is a collection of active and passive modules, where each active
module contains one or more Ada tasks. Although CubedOS is written in \SPARK\ there need not be
a one-to-one correspondence between CubedOS modules and \SPARK\ packages. In fact, modules are
routinely written as a collection of Ada packages in a package hierarchy. \pet{Provide a forward
  reference.}

Critical to the plug-and-play nature of CubedOS, each active module is self-contained and does
not make direct use of any code in any other active module, although passive modules serving as
library components can be used. All inter-module communication is done through the CubedOS
infrastructure with no direct sharing of data or executable content. In this respect CubedOS
modules are similar to processes in a conventional operating system. One consequence of this
policy is that a library used by several modules must be either duplicated in each module or
provided as an independent, passive module of its own. In this respect passive modules resemble
shared libraries in a conventional operating system and have similar concerns regarding task
safety and global data management.

In the language of operating systems, CubedOS can be said to have a microkernel architecture
where task management is provided by the Ada runtime system. Both low level facilities, such as
device drivers, and high level facilities, such as communication protocol handlers or navigation
algorithms, are all implemented as CubedOS modules. All modules are treated equally by CubedOS;
any layered structuring of the modules is imposed by programmer convention.

We are working on extending the architecture to allow module communication to occur between
modules in different operating system processes or even on different hosts. This extension will
allow CubedOS applications to be distributed across multiple spacecraft while still operating as
a single, integrated application. \pet{Provide a forward reference.}

% TODO: Some of this discussion, along with some diagrams, should perhaps be moved to the
% section describing the architecture of CubedOS. Some orientation to the system is appropriate
% in the introduction... but how much?

In addition to inter-module communication, CubedOS provides several general purpose facilities.
In some cases only the interface to the facility is described and different implementations are
possible (even encouraged). Having a standard interface allows other components of CubedOS to be
programmed against that interface without concern about the underlying implementation.

\begin{itemize}
\item An asynchronous message passing system with mailboxes. This, together with the underlying
  Ada runtime system constitutes the kernel of CubedOS.
\item A runtime library of useful packages, all verified with \SPARK.
\item A real time clock module.
\item A file system interface.
\item A radio communications interface.
\end{itemize}

A CubedOS system also requires drivers for the various hardware components in the system.
Although the specific drivers required will vary from mission to mission, CubedOS does provide a
general \newterm{driver model} that allows components to communicate with drivers fairly
generically. In a typical system there will be low level drivers for accessing hardware busses
as well as higher level drivers for sending/receiving commands from subsystems such as the
radio, the power system, the camera, etc. The low level drivers constitute the bulk of the
CubedOS LLAL.

CubedOS is currently a work in progress. It is our intention to release CubedOS as open source
once it has been further refined. We anticipate this to happen in late 2020.
